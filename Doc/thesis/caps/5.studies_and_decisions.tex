\chapter{Studies and Decisions} \label{cap:studies_and_decisions}

This chapter is dedicated to exploring the essential tools required for the
project's development and understanding their significance. The initial section
of this chapter focuses on delineating the functional and non-functional
requirements that the system must fulfill. Subsequently, the second part
identifies and elucidates the selected technologies for the project.

\section{System Requirement}

This section examines the essential prerequisites for melanoma detection. It is
crucial to differentiate between the final system and the developmental efforts
involved. The construction and training of a Convolutional Neural Network (CNN)
demand considerable computational power, sophisticated libraries, and often a
substantial amount of time.  \\

The requirements can be classified into two categories:

\begin{itemize}
  \item \textbf{Functional Requirements}

    Functional requirements pertain to the capabilities that the product must
    possess to fulfill specific user needs.

  \item \textbf{Non-functional Requirements}

    Non-functional requirements encompass aspects such as usability, performance,
    reliability.

\end{itemize}

\newpage

\subsection{Functional Requirements}

The functional requirements of the system are:

\begin{itemize} \item The objective is to classify whether a single image or a
      set of images, specifically from a dermoscopy, contains a melanoma or
      not. The classification task is binary, aiming to determine the presence
      or absence of melanoma.

    \item Expose models through an API that can classify whether an image or a
      set of images contains melanoma or not. The API should not only provide
      the classification results but also offer additional information such as
      the specific model used and the probabilities associated with the
      prediction..

    \item A user-friendly interface (UI) is implemented to facilitate loading
      of images and communicate with the API via the HTTP\footnote{Hypertext
      Transfer Protocol.} protocol for image inference. The UI should display
      the results of the selected model for the bulk of loaded images.

    \item Create images of each service, to distribute the services using container
      technology.
\end{itemize}

\subsection{Non-functional Requirements}

For training models, Jupyter Notebook, Python3, and Anaconda are used, ensuring
compatibility with any computer to run the code and classify images. Sufficient
RAM is required to handle the desired images effectively. \\

Since the API and UI services are containerized, any container management tool
such as Docker or Podman can be used, providing the necessary environment to
work with containers.

\newpage

\section{Hardware}

The hardware used for the thesis encompassed a range of machines. The primary
development machine employed was a laptop with limited performance.
Additionally, a machine provided by Google was utilized within the Google Colab
environment for training models that did not require high computational
capabilities. Furthermore, for a week of work at the VICOROB laboratory, a
separate machine was utilized specifically for training models with more
extensive data augmentation requirements, necessitating higher computational
resources. \\

Table \ref{table:dev-machine}, \ref{table:google-machine}, and \ref{table:vicorobot-machine} provide an overview of the characteristics of the machines employed for the project. Each of these machines had its own specific use, and it is recommended to select the an appropriate machine based on specific needs.

\begin{table}[H]
  \centering
  \begin{tabular}{cc}
    \toprule
    \multicolumn{2}{c}{\textbf{Development Machine}} \\
    \midrule
    OS & Fedora Linux 38 \\
    CPU & Intel i5-8250U \\
    GPU & Intel UHD Graphics 620 \\
    RAM & 8GB \\
    Disk & 225GB \\
    \bottomrule
  \end{tabular}
  \caption[Development Machine Metrics.]
  {\textit{Development Machine Metrics.
  This machine was only used for programming and searching thesis information.}}
  \label{table:dev-machine}
\end{table}


\begin{table}[H]
  \centering
  \begin{tabular}{cc}
    \toprule
    \multicolumn{2}{c}{\textbf{Google Machine}} \\
    \midrule
    OS & Ubuntu 20.04.6 LTS \\
    CPU & Intel Xeon \\
    GPU & Tesla T4, 16GB \\
    RAM & 12GB \\
    Disk & 125GB \\
    \bottomrule
  \end{tabular}
  \caption[Google Machine Metrics.]
  {\textit{Google Machine Metrics.
  Note that this machine was used to train models that take at most a pair of hours to accomplish the training.}}
  {\label{table:google-machine}}
\end{table}

\newpage

\begin{table}[H]
  \centering
  \begin{tabular}{cc}
    \toprule
    \multicolumn{2}{c}{\textbf{VICOROB Machine}} \\
    \midrule
    OS & Ubuntu 20.04.2 LTS \\
    CPU & Intel(R) Xeon(R) Silver \\
    GPU & A100, 80GB\\
    RAM & 396GB \\
    Disk & 6.3T \\
    \bottomrule
  \end{tabular}
  \caption[VICOROB Machine Metrics.]
  {\textit{VICOROB Machine Metrics.
  This machine was used to train models that required high computational resources because of data augmentation.}}
  {\label{table:vicorobot-machine}}
\end{table}

\section{Software}

In order to establish and operate the CAD infrastructure, it was crucial to
establish separate environments for creating the services and training the
models, ensuring that all the required packages were included. \\

The following pages will outline and provide detailed descriptions of each tool
utilized throughout the thesis. These tools played a crucial role in various
aspects, including the training process, creation of services such as user
interface (UI) and application programming interface (API), deployment of these
services with trained models, management of the source code, and comprehensive
documentation. Each tool's significance and contribution to the thesis will be
thoroughly explained, shedding light on their specific functionalities and
importance in the research project.

\begin{itemize}
  \item \textbf{Python 3.9}

    Python  is a high-level, interpreted
    programming language known for its simplicity and readability. It was
    created by Guido van Rossum and released in 1991. Its design philosophy
    emphasizes code readability with the use of significant indentation. \\

    Python is an ideal choice for machine learning and AI-based projects due to
    several advantages. These include its simplicity and consistency, access to
    excellent libraries and frameworks specifically designed for AI and ML,
    flexibility, platform independence, and a large and supportive community.

    \newpage

    \begin{figure}[H]
      \centering
      \includegraphics[width=0.15\textwidth]{imatges/studies_and_decisions/python-logo-only.png}
      \caption[Python Logo]{\textit{Python Logo. Illustration by Python Software Foundation}}
      {\label{fig:python-logo}}
    \end{figure}

  \item \textbf{Anaconda}

    Anaconda  is a widely used open-source distribution of the Python programming language. It simplifies the management and deployment of data science and machine learning environments. With Anaconda, you can easily install and manage Python packages using the conda package management system. It also provides tools for creating isolated environments to ensure consistent dependencies and package versions.

    \begin{figure}[H]
      \centering
      \begin{adjustbox}{trim=0cm 1cm 0cm 1cm, clip}
        \includegraphics[width=0.5\textwidth]{imatges/studies_and_decisions/anaconda-logo.png}
      \end{adjustbox}
      \caption[Anaconda Logo]{\textit{Anaconda Logo. Illustration by Anaconda}}
      {\label{fig:anaconda-logo}}
    \end{figure}

  \item \textbf{Jupyter Notebook}

    Jupyter Notebook  is an open-source
    web application that allows to create and share interactive documents
    containing live code, visualizations, and text. It provides an environment
    for data analysis, visualization, and prototyping. With Jupyter Notebooks,
    you can write and execute code in individual cells, making it easy to
    experiment and iterate. It supports multiple programming languages such as
    Python, R and Haskell.

    \begin{figure}[H]
      \centering
      \includegraphics[width=0.2\textwidth]{imatges/studies_and_decisions/jupyter-notebook.png}
      \caption[Jupyter Notebook Logo]{\textit{Jupyter Notebook Logo. Illustration by Jupyter.org}}
      {\label{fig:jupyter-logo}}
    \end{figure}

    \newpage

  \item \textbf{CUDA}

    CUDA, which stands for Compute
    Unified Device Architecture, is a parallel computing platform and
    programming model developed by NVIDIA. It enables developers to leverage
    the power of NVIDIA GPUs (Graphics Processing Units) for high-performance
    computing tasks. \\

    CUDA has gained widespread adoption in the scientific and computational
    research communities due to its ability to leverage the computational power
    of GPUs. It has become an essential tool for accelerating a wide range of
    applications, from physics simulations and computational biology to data
    analytics and artificial intelligence.

    \begin{figure}[H]
      \centering
      \begin{adjustbox}{trim=0cm 0.25cm 0cm 0.5cm, clip}
        \includegraphics[width=0.325\textwidth]{imatges/studies_and_decisions/nvidia-cuda.jpg}
      \end{adjustbox}
      \caption[CUDA Logo]{\textit{CUDA Logo. Illustration by Nvidia Corporation}}
      {\label{fig:cuda-logo}}
    \end{figure}

    \vspace{0.5cm}
    \textbf{PyTorch}

    PyTorch  is an open-source machine
    learning framework widely used for deep learning tasks. It provides a
    flexible and dynamic approach to building and training neural networks.
    PyTorch stands out for its integration of GPU acceleration, enabling
    efficient computations using NVIDIA GPUs. \\

    PyTorch offers a high-level API called torchvision, which simplifies common
    computer vision tasks such as image classification, object detection, and
    image generation. It also integrates with other libraries and tools in the
    Python ecosystem, making it easy to combine PyTorch with popular frameworks
    like NumPy, SciPy, and pandas.

    \begin{figure}[H]
      \centering
      \includegraphics[width=0.325\textwidth]{imatges/studies_and_decisions/pytorch-logo.png}
      \caption[PyTorch Logo]{\textit{PyTorch Logo. Illustration by PyTorch.org}}
      {\label{fig:pytorch-logo}}
    \end{figure}

  \item \textbf{OpenCV}

    OpenCV (Open Source Computer Vision Library), is a popular open-source
    computer vision and image processing library. It provides a wide range of
    functions and algorithms for tasks such as image and video processing,
    object detection and tracking, feature extraction, and more. \\

    OpenCV  allows developers to read,
    write, and manipulate images and videos efficiently. The library offers
    various image processing functions, including filtering, resizing, color
    conversion, and geometric transformations.

    \begin{figure}[H]
      \centering
      \begin{adjustbox}{trim=1cm 1cm 1cm 1cm, clip}
        \includegraphics[width=0.7\textwidth]{imatges/studies_and_decisions/OpenCV-logo.png}
      \end{adjustbox}
      \caption[OpenCV Logo]{\textit{OpenCV Logo. Illustration by OpenCV Team}}
      {\label{fig:opencv-logo}}
    \end{figure}

  \item \textbf{Albumentations}

    Albumentations  is an
    open-source Python library for image augmentation in machine learning and
    computer vision tasks. It provides a wide range of transformations and
    optimizations to enhance training data, improving the performance and
    generalization of machine learning models. With high-performance
    implementation and seamless integration with popular frameworks,
    Albumentations is widely used for efficient and customizable image
    augmentation.

    \begin{figure}[H] \centering
      \includegraphics[width=0.13\textwidth]{imatges/studies_and_decisions/albumentations-logo.png}
      \caption[Albumentations Logo]{\textit{Albumentations Logo. Illustration
      by Albumentations Team}} {\label{fig:albumentations-logo}} \end{figure}

    \item \textbf{NumPy}

    NumPy  is a Python library for
    efficient numerical computing. It provides a powerful array object for
    manipulation and computation of multidimensional arrays. With optimized
    operations and support for broadcasting.

    \begin{figure}[H]
      \centering
      \includegraphics[width=0.225\textwidth]{imatges/studies_and_decisions/numpy-logo.png}
      \caption[NumPy Logo]{\textit{NumPy Logo. Illustration by NumPy Organazation}}
      {\label{fig:numpy-logo}}
    \end{figure}

  \item \textbf{Pandas}

    Pandas  is a popular Python library
    for data manipulation and analysis. It provides data structures and
    functions to efficiently work with structured data, such as tabular data
    and time series. Pandas' key features include its DataFrame object, which
    allows for easy handling of data, indexing, filtering, and aggregation
    operations. With its intuitive and powerful functionality, Pandas is widely
    used in data preprocessing, exploration, and analysis tasks.

    \begin{figure}[H]
      \centering
      \includegraphics[width=0.3\textwidth]{imatges/studies_and_decisions/pandas-logo.png}
      \caption[Pandas Logo]{\textit{Pandas Logo. Illustration by The pandas development team}}
      {\label{fig:pandas-logo}}
    \end{figure}


  \item \textbf{Matplotlib}

    Matplotlib  is a widely-used
    Python library for creating static, animated, and interactive
    visualizations. It provides a flexible and comprehensive range of functions
    and tools for generating plots, charts, histograms, and more. Matplotlib
    allows customization of various aspects of visualizations, including
    colors, labels, titles, axes, and legends. With its extensive functionality
    and compatibility with NumPy arrays, Matplotlib is a go-to library for data
    visualization and presentation in Python.

    \begin{figure}[H]
      \centering
      \includegraphics[width=0.45\textwidth]{imatges/studies_and_decisions/matplotlib-logo.png}
      \caption[Matplotlib Logo]{\textit{Matplotlib Logo. Illustration by Matplotlib Organization}}
      {\label{fig:matplotlib-logo}}
    \end{figure}

    \newpage

  \item \textbf{Seaborn}

    Seaborn  is a Python data visualization library built on top of Matplotlib. It provides a high-level interface for creating attractive and informative statistical graphics. Seaborn simplifies the process of generating complex visualizations by offering a range of built-in functions for creating appealing plots, such as scatter plots, histograms, bar plots, and heatmaps. Seaborn is widely used for exploratory data analysis and presentation of statistical insights.

    \begin{figure}[H]
      \centering
      \includegraphics[width=0.25\textwidth]{imatges/studies_and_decisions/seaborn-logo.png}
      \caption[Seaborn Logo]{\textit{Seaborn Logo. Illustration by Seaborn Organization}}
      {\label{fig:seaborn-logo}}
    \end{figure}

  \item \textbf{WandB}

    Weights \& Biases  is a machine learning
    experiment tracking and visualization platform. It offers a suite of tools
    to track, organize, and analyze machine learning experiments. With wandb,
    developers can log metrics, hyper-parameters, and model checkpoints during
    training, making it easy to compare and analyze different experiments.

    \begin{figure}[H]
      \centering
      \includegraphics[width=0.4\textwidth]{imatges/studies_and_decisions/wandb-logo.png}
      \caption[Seaborn Logo]{\textit{Seaborn Logo. Illustration by Weights \& Biases Inc}}
      {\label{fig:wandb-logo}}
    \end{figure}

  \item \textbf{FastAPI}

    FastAPI  is a modern,
    high-performance web framework for building APIs with Python. It emphasizes
    simplicity, speed, and type annotations to create efficient and scalable
    web applications. FastAPI leverages Python's type hints for automatic
    request and response validation, enabling faster development and reducing
    the chances of introducing bugs. It provides asynchronous capabilities
    using Python's asyncio framework, allowing for high-concurrency and
    efficient handling of multiple requests. With its intuitive API declaration
    syntax and automatic generation of interactive documentation, FastAPI
    simplifies the process of building robust and well-documented APIs.

    \begin{figure}[H]
      \centering
      \begin{adjustbox}{trim=0cm 0.25cm, clip}
        \includegraphics[width=0.35\textwidth]{imatges/studies_and_decisions/fastapi-logo.png}
      \end{adjustbox}
      \caption[FastAPI Logo]{\textit{FastAPI Logo. Illustration by tiangolo}}
      {\label{fig:fastapi-logo}}
    \end{figure}

  \item \textbf{JS}

    JavaScript  is a versatile programming language primarily used for client-side web development. It enables dynamic and interactive functionality on websites, allowing for real-time manipulation of HTML elements, handling user events, and modifying web page content.

    \begin{figure}[H]
      \centering
      \begin{adjustbox}{trim=0cm 0cm, clip}
        \includegraphics[width=0.15\textwidth]{imatges/studies_and_decisions/js-logo.png}
      \end{adjustbox}
      \caption[JS Logo]{\textit{JS Logo. Illustration by JS Organization}}
      {\label{fig:js-logo}}
    \end{figure}

  \item \textbf{SvelteKit}

    SvelteKit  is a high-performance
    JavaScript framework for building web applications. It uses a
    compiler-based approach, resulting in optimized applications with minimal
    overhead. It provides built-in routing, supports Server-Side Rendering
    (SSR) and Static Site Generation (SSG), and promotes component reusability.
    SvelteKit offers a developer-friendly environment, integrates well with
    APIs, and generates small bundles for faster loading times.

    \begin{figure}[H]
      \centering
      \begin{adjustbox}{trim=0cm 0cm, clip}
        \includegraphics[width=0.12\textwidth]{imatges/studies_and_decisions/sveltekit-logo.png}
      \end{adjustbox}
      \caption[SvelteKit Logo]{\textit{SvelteKit Logo. Illustration by WIKIPEDIA}}
      {\label{fig:sveltekit-logo}}
    \end{figure}

  \item \textbf{Docker}

    Docker  is an open-source platform
    for building and deploying applications in isolated containers. It
    simplifies application management, ensures portability across different
    environments, and provides isolation for enhanced scalability and security.
    With Docker, developers can package applications and their dependencies
    into lightweight, self-contained containers, making deployment and scaling
    more efficient.

    \begin{figure}[H]
      \centering
      \begin{adjustbox}{trim=0cm 0cm, clip}
        \includegraphics[width=0.4\textwidth]{imatges/studies_and_decisions/docker-logo.png}
      \end{adjustbox}
      \caption[SvelteKit Logo]{\textit{SvelteKit Logo. Illustration by WIKIPEDIA}}
      {\label{fig:docker-logo}}
    \end{figure}


  \item \textbf{Podman}

    Podman  is designed to manage
    containers and container images, offering features similar to Docker but
    with a focus on security and compatibility with industry standards. It
    allows you to build, run, and manage containers without the need for a
    separate daemon. \\

    One key advantage of Podman is its rootless mode, which allows users to run
    containers as non-root, enhancing security and isolation. This eliminates
    the need for privileged access, making Podman a preferred choice for
    environments where running containers as non-root is required.

    \begin{figure}[H]
      \centering
      \begin{adjustbox}{trim=0cm 0.25cm, clip}
        \includegraphics[width=0.3\textwidth]{imatges/studies_and_decisions/podman-logo.png}
      \end{adjustbox}
      \caption[Podman Logo]{\textit{Podman Logo. Illustration by Red Hat}}
      {\label{fig:podman-logo}}
    \end{figure}

  \item \textbf{Git}

    Git  is a distributed version control
    system used for tracking changes in software projects. It enables
    collaboration, allows for branching and merging, handles large projects
    efficiently, and provides robust backup options.

    \begin{figure}[H]
      \centering
      \begin{adjustbox}{trim=0cm 0.25cm, clip}
        \includegraphics[width=0.25\textwidth]{imatges/studies_and_decisions/git-logo.jpg}
      \end{adjustbox}
      \caption[Git Logo]{\textit{Git Logo. Illustration by Git Community}}
      {\label{fig:git-logo}}
    \end{figure}

  \item \textbf{GitHub \& GitLab}

    Both GitHub and GitLab are used as centralized platforms for hosting and
    managing Git repositories.

    \begin{figure}[H]
      \centering
      \begin{adjustbox}{trim=0cm 0cm, clip}
        \includegraphics[width=0.15\textwidth]{imatges/studies_and_decisions/github-mark.png}
      \end{adjustbox}
      \caption[GitHub Logo]{\textit{GitHub Logo. Illustration by GitHub}}
      {\label{fig:github-logo}}
    \end{figure}

    \begin{figure}[H]
      \centering
      \begin{adjustbox}{trim=0cm 0cm, clip}
        \includegraphics[width=0.15\textwidth]{imatges/studies_and_decisions/gitlab-logo.png}
      \end{adjustbox}
      \caption[GitLab Logo]{\textit{GitLab Logo. Illustration by GitLab}}
      {\label{fig:gitlab-logo}}
    \end{figure}

  \item \textbf{cURL}

    cURL is a command-line tool and library used to transfer data with URLs\footnote{Uniform Resource Locator.}.
    It supports various protocols (HTTP, HTTPS, FTP, etc.) and is commonly used
    to make requests to web servers and retrieve or send data from and to
    different sources.

    \begin{figure}[H]
      \centering
      \begin{adjustbox}{trim=0cm 0cm, clip}
        \includegraphics[width=0.25\textwidth]{imatges/studies_and_decisions/curl-logo.png}
      \end{adjustbox}
      \caption[cURL Logo]{\textit{cURL Logo. Illustration by curl.se}}
      {\label{fig:curl-logo}}
    \end{figure}

  \item \textbf{LaTeX}

    LaTeX is a typesetting system used for creating high-quality documents,
    particularly those with mathematical equations and scientific content. It
    focuses on the structure and formatting of documents, allowing users to
    concentrate on content.

    \begin{figure}[H]
      \centering
      \begin{adjustbox}{trim=0cm 0cm, clip}
        \includegraphics[width=0.25\textwidth]{imatges/studies_and_decisions/latex-logo.png}
      \end{adjustbox}
      \caption[LaTeX Logo]{\textit{LaTeX Logo. Illustration by LaTeX}}
      {\label{fig:latex-logo}}
    \end{figure}
\end{itemize}
