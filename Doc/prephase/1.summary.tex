\chapter*{Summary}
%\label{cap:resum}

Skin cancer, including melanoma, is a significant global public health concern.
Melanoma presents a considerable challenge due to its high mortality rate and
the critical importance of early detection for successful treatment. Cancer
begins when healthy cells undergo changes that cause them to grow and divide
uncontrollably, forming tumors. These tumors can be classified as either
cancerous (malignant) or non-cancerous (benign). Malignant tumors have the
potential to invade nearby tissues and spread to other parts of the body
through metastasis. \\

In recent times, there has been a growing focus on automating tasks in the
medical field through Computer-Aided Diagnosis (CAD). However, integrating CAD
into the medical system remains a significant challenge. \\

The objective of this project is to explore the current state of deep learning
techniques in image detection and assess their applicability and reliability in
creating a real-time system capable of classifying melanoma images. \\

The investigation led to the training of eight different models based on
convolutional layers with pre-trained weights from ResNet18 in the ImageNet
dataset. These models were distinguished by applying various machine learning
techniques during the training process, including regularization, data
augmentation, and learning rate scheduling, along with hyper-parameter
configuration. \\

The resulting models were trained on an unbalanced dataset of eight different
classes, comprising twenty-five thousand real dermoscopy images. Validation was
performed using three thousand different samples during the training phase. To
evaluate the models, a holdout set scheme was used, where they were tested on
unseen data, using three thousand samples, achieving good results (Table
\ref{table:test-set-resume-metrics}). To validate and further test the models,
Test-Time augmentation was employed to emulate an ensemble of models on the
fly. \\

This project goes beyond pure investigation and includes engineering efforts to
make sophisticated knowledge accessible to a wider audience. A CAD
infrastructure, based on micro-services, was created to provide medical
professionals with a powerful tool for advanced diagnostic capabilities. This
infrastructure integrates the trained models with an API and a user interface
(UI), enabling professionals to improve patient care and outcomes. \\

The developed CAD infrastructure stands as a significant advancement in the
automation of melanoma detection, enhancing the potential for early diagnosis
and better patient outcomes. \\

The research's final conclusions can be summarized in two key points. Firstly,
there are significant benefits in developing platforms and infrastructure that
provide accessible machine learning solutions. Secondly, it is crucial to be
cautious when implementing certain machine learning solutions, as some "solved"
problems may raise moral and ethical concerns. \\

This project was carried out in partnership with the VICOROB
research group at the University of Girona and Accenture S.L. The documentation
and CAD infrastructure, along with the training and infrastructure code, are
freely accessible on GitHub at the following URL:
\url{https://github.com/wilberquito/melanoma.thesis}.
