\section{Manual Installation}
\label{appendix:appendix_a}

We will provide you with the necessary tools to install the CAD infrastructure.
You won't have to worry about managing the installation of Python packages and NPM packages,
as we have created images that include all the required packages for seamless execution. \\

Although you need some tools for the installation process,
these tools are specifically for building the project and are independent of the services created. \\

This example employs Fedora Linux 38 as the operating system; however,
all the tools used for building the platform are cross-platform, ensuring smooth execution on other operating systems as well. \\

The tools you will need for the installation are these:

\begin{itemize}
  \item Git
  \item Docker or Podman
  \item docker-compose
  \item CURL
\end{itemize}

To install the project you need to be super-user.
Using the a Linux SO command line, you do:

\begin{Verbatim}[fontsize=\small]
sudo su
\end{Verbatim}

Once you are in super-user, you need to downlaod
and execute the bash script that we provide in this master thesis, this script is in
charge of downlaoding the thesis source code and thesis
assets (trained models and configurations), this script is also
in charge of creating the services images and wake up the
the corresponding containers from the images. \\

If you are using Docker as main container tool:

\begin{Verbatim}[fontsize=\small]
$ curl https://gitlab.com/wilberquito/open.thesis/-/raw/main/Make.sh | bash
\end{Verbatim}

If you use Podman as main container tool:

\begin{Verbatim}[fontsize=\small]
$ alias docker=podman && \
  curl https://gitlab.com/wilberquito/open.thesis/-/raw/main/Make.sh | bash
\end{Verbatim}

\newpage

To verify that the services were wake up, you can consult it using the following
command instruction:

\\

\begin{Verbatim}[fontsize=\small]
$ docker ps
\end{Verbatim}
