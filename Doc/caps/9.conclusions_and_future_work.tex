\chapter{Conclusions and Future Work}
\label{cap:concl}


The objective of this project was to employ deep learning methods to train multiple models for detecting melanoma in an unbalanced dataset of dermoscopy images.
Once these models were trained, they were exposed through a microservice architecture.
The architecture consisted of two microservices: an API that exposed the models and a UI for interacting with the API.
To achieve this, eight different models were trained, all utilizing transfer learning with the ResNet-18 pre-trained model as the base.
Some of the proposed models involved modifying only the fully-connected layer of ResNet-18 to classify between eight classes,
while others utilized more complex fully connected layers with Dropout. \\

For training these eight models,
various hyperparameters and learning policies using schedulers were experimented with.
The models can be categorized into two groups: those trained with additional regularization,
such as data augmentation and dropout layers, and those without such extra regularization.
The second group of models was trained for only a few epochs, as it was quickly observed that they tended to overfit. \\

\begin{table}[H]
\centering
\begin{tabular}{lclc}
    \toprule
 Model &  Test AUC & \cellcolor{gray!50}Model & \cellcolor{gray!50}Test AUC  \\
\midrule
 M0 & 0.892 & \cellcolor{gray!50}M4 & \cellcolor{gray!50}0.858 \\
 M1 $\star$ & 0.892 & \cellcolor{gray!50}M5 $\star$ & \cellcolor{gray!50}0.843 \\
 M2 $\ast$ &  0.885 &  \cellcolor{gray!50}M6 $\ast$ & \cellcolor{gray!50}0.848 \\
 M3 $\bullet$ & 0.886 & \cellcolor{gray!50}M7 $\bullet$ & \cellcolor{gray!50}0.849 \\
 \midrule
Mean &  88.875\% & \cellcolor{gray!50}Mean & \cellcolor{gray!50}84.950\%  \\
SD &  0.377\%  &   \cellcolor{gray!50}SD &  \cellcolor{gray!50}0.625\%  \\

\bottomrule
\end{tabular}
\caption[Metrics in Test Dataset]
  {\textit{Metrics in Test Dataset. Table by Author}}
{\label{table:test-set-resume-metrics}}
\end{table}
