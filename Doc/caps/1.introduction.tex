\chapter{Introduction}
\label{cap:intro}

\section{Problem Statement}

Skin cancer, including melanoma, represents a significant public health concern
worldwide. Melanoma, in particular, poses a considerable challenge due to its
high mortality rate and the need for early detection for successful treatment.
Early and correct diagnosis is key for ensuring patients have the best possible
prognosis. Misdiagnosis of melanoma leads to more pathology and dermatology
malpractice claims than any cancer except breast cancer, as an early
misdiagnosis can greatly reduce a patient's survival chances \cite{Melanoma}. \\

Dermoscopy, also known as dermatoscopy (Figure \ref{fig:procedure_dermoscopy}),
is a noninvasive technique widely utilized for the examination of cutaneous
lesions. It involves the use of a handheld instrument called a dermatoscope to
visualize subsurface skin structures that are typically not visible to the
naked eye. The dermatoscope illuminates the skin surface and provides
magnification, allowing for a detailed examination of the epidermis, the
dermoepidermal junction, and the papillary dermis. By analyzing these
structures, dermatologists can identify specific features and patterns
associated with various skin conditions, including melanoma \cite{Dermoscopy}.

\begin{figure}[htb] \centering
  \includegraphics[width=6cm]{imatges/introduction/medical_procedure_dermastocopy.jpeg}
  \caption[Dermoscopy Procedure]{\textit{Dermoscopy Procedure. Illustration by MD Anderson Cancer Center}}
  {\label{fig:procedure_dermoscopy}}
\end{figure}

Worldwide, in 2020 an estimated 324,635 people were diagnosed with melanoma and
an estimated 57,043 people worldwide died from melanoma the same year
\cite{CancerStats}. The introduction of sophisticated machinery and new
techniques in dermoscopy procedures (Figure \ref{fig:subset_isic}) seems not
enough to fight against melanoma, but the developments in artificial
intelligence (AI) especially in deep learning techniques, have made
Computer-Aided Diagnosis (CAD) a promising path towards
medical automation.

\begin{figure}[H] \centering
  \begin{subfigure}{0.3\textwidth}
    \includegraphics[width=\textwidth]{imatges/introduction/subset_isic/ISIC_1752943.jpg}
  \end{subfigure}
  \hfill
  \begin{subfigure}{0.3\textwidth}
    \includegraphics[width=\textwidth]{imatges/introduction/subset_isic/ISIC_1766619.jpg}
  \end{subfigure}
  \hfill
  \begin{subfigure}{0.3\textwidth}
    \includegraphics[width=\textwidth]{imatges/introduction/subset_isic/ISIC_1448526.jpg}
  \end{subfigure}
  \caption[Dermoscopy Images]{\textit{Dermoscopy Images. Illustration by ISIC Archive}}
  \label{fig:subset_isic}
\end{figure}

However, there are several limitations that raise doubts about the
effectiveness of automated melanoma cancer classifiers and their suitability
for integration into the medical system. Firstly, certain methods are
constructed based on theoretical models of melanoma appearance, which may
restrict their applicability to specific morphologies and fail to capture the
wide range of variations seen in real-world scenarios. Secondly, AI systems
utilized in these classifiers are trained to address a singular and narrow
task. Unlike human dermatologists, these systems lack the ability to consider
holistic patient information when formulating a final diagnosis, reflecting the
concept of weak AI \cite{WeakAI}. Lastly, numerous methods have been trained
and evaluated using high-quality image frames, which may result in instability
when applied under real-time conditions where image quality is often
compromised. A fundamental part of machine learning is the problem of
generalization, that is, how to make sure that a trained model performs well on
unseen data. If the unseen data has different distribution, i.e. a domain shift
exists, the problem is significantly more difficult; even the smallest changes
in the statistics as compared to the training data can cause a deep neural
network to fail completely \cite{DomainShift}. \\

Addressing these limitations and developing melanoma cancer classifiers that
encompass a wider range of morphologies, incorporate holistic patient
information, and demonstrate robustness in real-world scenarios are crucial for
improving the reliability and effectiveness of melanoma detection and diagnosis
systems.

\section{Project Objectives}

The main objective of this project is to create a health care infrastructure,
focused on melanoma detection using deep learning methods to train a system
capable of detecting melanoma on dermoscopy images to test the ability of
computer-assisted image analysis. To this end, the gradual achievements that
must be accomplished are:

\begin{itemize}
  \item Gaining a comprehensive understanding of the theory
    behind deep learning and its practical applications.
  \item Analyzing images
    from dermoscopy and acknowledge its most important features.
  \item Train
    deep learning models with different techniques based on tranfer-learning,
    exploiting images of the melanoma ISIC \footnote{International Skin Imaging
    Collaboration, an international effort to improve melanoma diagnosis.}
    Challenge \cite{IsicChallenge}.
  \item Developing a CAD infrastructure. The CAD infrastructure, should contain
    the already trained models with a simple web UI an API and finally a
    mechanism using Docker to create the images of the services making it ease
    to deploy in any based Linux System.
\end{itemize}

\section{Personal Motivation}

I envisioned this project as a unique fusion of three personal passions.
Firstly, I was fueled by a deep fascination with human cognition and reasoning.
Machines, in my eyes, represented a novel paradigm through which I could delve
further into this captivating realm. \\

I am also motivated by the remarkable problem-solving capacity of data.
Regardless of its structure, data holds immense potential to uncover hidden
patterns, provide insights, and drive innovation. The ability to extract
meaningful information from data, regardless of its form, inspires me to
constantly expand my knowledge and skills in order to contribute to the field
of data science and make a tangible difference in the world. \\

Last but not least, I am driven by the immense power of automation and its
ability to democratize access to research knowledge. I am amazed by how
automation processes can extract value and make them readily available to
professionals and the public alike.

\section{Statement of Originality}

I, Wilber Eduardo Bermeo Quito, declare that the thesis titled "A Platform for
Classifying Melanoma" is an original work completed with the support and
collaboration of Accenture SL and the VICOROB research group. \\

The content presented in this thesis is the outcome of my independent research
efforts, guided by the knowledge and expertise acquired through my academic
studies and the valuable contributions from Accenture S.L and the VICOROB
research group. \\

I acknowledge the importance of academic integrity and the consequences of
plagiarism. Hence, I affirm that all the information, data, results, figures,
and conclusions presented in this thesis are authentic and original. Any
references or sources used have been appropriately cited and referenced.

\section{Regulatory Framework}

The inclusion of legal considerations has become a significant aspect of the
field of medical imaging. Privacy concerns and the potential misuse of personal
information make sharing and distributing medical data particularly
challenging. To address these limitations, recent research collaborations have
focused on promoting the sharing of patient data through de-identification
methods. However, it is crucial to thoroughly analyze the obligations related
to the protection of individuals and their personal data before engaging in
projects involving medical imaging. \\

When working with medical images, it is the utmost importance to prioritize
patient privacy rights. In the context of developing a skin lesions database,
it is necessary to obtain signed consent from patients for the publication of
their data. For this thesis, the ISIC Archive database was utilized, this
database serves as a publicly accessible resource for teaching, research, and
the development and testing of diagnostic artificial intelligence algorithms,
and it resolves any concerns related to consent \cite{IsicArchive}. It is a
large and continually expanding open-source archive of skin images.

\section{Ethical Concern}

The outcome of this thesis is a melanoma classification tool, developed using
"black box" models. Although these well-trained models usually yield highly
accurate results, their lack of transparency poses a challenge in terms of
explainability. \\

The issue of explainability is a significant concern in this study since the
possibility of misclassification is considerable. When encountering a false
negative, it becomes crucial to understand the reasons behind the
misclassification. Consequently, both the introduction and conclusion of the
thesis emphasize that this tool is intended to assist human decision-making
rather than being an autonomous decision-making system.

\section{Contribution to Melanoma Detection}

In this section, we present our contribution to the field of melanoma
detection, which draws inspiration from related research in the development of
a melanoma CAD (Computer-Aided Diagnosis) infrastructure classifier. This
thesis is enriched by incorporating ideas from existing works and
state-of-the-art projects, which are elaborated in Chapter \ref{cap:estat}. \\

The thesis comprises multiple models employing various techniques, including
transfer learning, data augmentation, just-in-time testing, regularization, and
others. To compare these models, tools like W\&B (Weight \& Biases), a MLOps
platform for experiment tracking, and MLXTEND, providing utilities and
extensions for machine learning and data science in Python's scientific
computing stack, were utilized. The experiments were conducted using the ISIC
Archive dataset, consisting of benign and malignant skin lesions. Prior to
analysis, the dataset underwent preprocessing to enhance quality and normalize
features. Consequently, accurate classification and differentiation of melanoma
lesions from benign ones were achieved. \\

In order to to ensure the usability of the trained models,
two services were developed. These services include a user-friendly UI and an
API, both of which were containerized using Docker technology. By
containerizing the UI, an intuitive and interactive user experience was
provided, enabling users to seamlessly interact with the melanoma CAD system.
Similarly, containerizing the API streamlined request handling and prediction
serving, resulting in efficient performance. This approach facilitated easy
deployment across various platforms.
